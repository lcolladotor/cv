%% start of file `template.tex'.
%% Copyright 2006-2013 Xavier Danaux (xdanaux@gmail.com).
%
% This work may be distributed and/or modified under the
% conditions of the LaTeX Project Public License version 1.3c,
% available at http://www.latex-project.org/lppl/.


\documentclass[11pt,letterpaper,sans]{moderncv}\usepackage[]{graphicx}\usepackage[]{color}
%% maxwidth is the original width if it is less than linewidth
%% otherwise use linewidth (to make sure the graphics do not exceed the margin)
\makeatletter
\def\maxwidth{ %
  \ifdim\Gin@nat@width>\linewidth
    \linewidth
  \else
    \Gin@nat@width
  \fi
}
\makeatother

\definecolor{fgcolor}{rgb}{0.345, 0.345, 0.345}
\newcommand{\hlnum}[1]{\textcolor[rgb]{0.686,0.059,0.569}{#1}}%
\newcommand{\hlstr}[1]{\textcolor[rgb]{0.192,0.494,0.8}{#1}}%
\newcommand{\hlcom}[1]{\textcolor[rgb]{0.678,0.584,0.686}{\textit{#1}}}%
\newcommand{\hlopt}[1]{\textcolor[rgb]{0,0,0}{#1}}%
\newcommand{\hlstd}[1]{\textcolor[rgb]{0.345,0.345,0.345}{#1}}%
\newcommand{\hlkwa}[1]{\textcolor[rgb]{0.161,0.373,0.58}{\textbf{#1}}}%
\newcommand{\hlkwb}[1]{\textcolor[rgb]{0.69,0.353,0.396}{#1}}%
\newcommand{\hlkwc}[1]{\textcolor[rgb]{0.333,0.667,0.333}{#1}}%
\newcommand{\hlkwd}[1]{\textcolor[rgb]{0.737,0.353,0.396}{\textbf{#1}}}%

\usepackage{framed}
\makeatletter
\newenvironment{kframe}{%
 \def\at@end@of@kframe{}%
 \ifinner\ifhmode%
  \def\at@end@of@kframe{\end{minipage}}%
  \begin{minipage}{\columnwidth}%
 \fi\fi%
 \def\FrameCommand##1{\hskip\@totalleftmargin \hskip-\fboxsep
 \colorbox{shadecolor}{##1}\hskip-\fboxsep
     % There is no \\@totalrightmargin, so:
     \hskip-\linewidth \hskip-\@totalleftmargin \hskip\columnwidth}%
 \MakeFramed {\advance\hsize-\width
   \@totalleftmargin\z@ \linewidth\hsize
   \@setminipage}}%
 {\par\unskip\endMakeFramed%
 \at@end@of@kframe}
\makeatother

\definecolor{shadecolor}{rgb}{.97, .97, .97}
\definecolor{messagecolor}{rgb}{0, 0, 0}
\definecolor{warningcolor}{rgb}{1, 0, 1}
\definecolor{errorcolor}{rgb}{1, 0, 0}
\newenvironment{knitrout}{}{} % an empty environment to be redefined in TeX

\usepackage{alltt}        % possible options include font size ('10pt', '11pt' and '12pt'), paper size ('a4paper', 'letterpaper', 'a5paper', 'legalpaper', 'executivepaper' and 'landscape') and font family ('sans' and 'roman')

% moderncv themes
\moderncvstyle{casual}                             % style options are 'casual' (default), 'classic', 'oldstyle' and 'banking'
\moderncvcolor{blue}                               % color options 'blue' (default), 'orange', 'green', 'red', 'purple', 'grey' and 'black'
%\renewcommand{\familydefault}{\sfdefault}         % to set the default font; use '\sfdefault' for the default sans serif font, '\rmdefault' for the default roman one, or any tex font name
%\nopagenumbers{}                                  % uncomment to suppress automatic page numbering for CVs longer than one page

% character encoding
\usepackage[utf8]{inputenc}                       % if you are not using xelatex ou lualatex, replace by the encoding you are using
%\usepackage{CJKutf8}                              % if you need to use CJK to typeset your resume in Chinese, Japanese or Korean

% adjust the page margins
\usepackage[scale=0.75]{geometry}
%\setlength{\hintscolumnwidth}{3cm}                % if you want to change the width of the column with the dates
%\setlength{\makecvtitlenamewidth}{10cm}           % for the 'classic' style, if you want to force the width allocated to your name and avoid line breaks. be careful though, the length is normally calculated to avoid any overlap with your personal info; use this at your own typographical risks...

% personal data
\name{Leonardo}{Collado-Torres}
%\title{Resumé title}                               % optional, remove / comment the line if not wanted
\address{615 N. Wolfe Street, Room E3032}{21205-2179}{United States}% optional, remove / comment the line if not wanted; the "postcode city" and and "country" arguments can be omitted or provided empty
%\phone[mobile]{+1~(234)~567~890}                   % optional, remove / comment the line if not wanted
\phone[fixed]{+1~(410)~955~0958}                    % optional, remove / comment the line if not wanted
%\phone[fax]{+3~(456)~789~012}                      % optional, remove / comment the line if not wanted
\email{lcolladotor@gmail.com}                               % optional, remove / comment the line if not wanted
\homepage{http://www.biostat.jhsph.edu/$\sim$lcollado/}                         % optional, remove / comment the line if not wanted
\extrainfo{Blog: http://lcolladotor.github.io/}                 % optional, remove / comment the line if not wanted
%\photo[64pt][0.4pt]{picture}                       % optional, remove / comment the line if not wanted; '64pt' is the height the picture must be resized to, 0.4pt is the thickness of the frame around it (put it to 0pt for no frame) and 'picture' is the name of the picture file
%\quote{Some quote}                                 % optional, remove / comment the line if not wanted

% to show numerical labels in the bibliography (default is to show no labels); only useful if you make citations in your resume
%\makeatletter
%\renewcommand*{\bibliographyitemlabel}{\@biblabel{\arabic{enumiv}}}
%\makeatother
%\renewcommand*{\bibliographyitemlabel}{[\arabic{enumiv}]}% CONSIDER REPLACING THE ABOVE BY THIS

% bibliography with mutiple entries
%\usepackage{multibib}
%\newcites{book,misc}{{Books},{Others}}
%----------------------------------------------------------------------------------
%            content
%----------------------------------------------------------------------------------
\IfFileExists{upquote.sty}{\usepackage{upquote}}{}
\begin{document}
%\begin{CJK*}{UTF8}{gbsn}                          % to typeset your resume in Chinese using CJK
%-----       resume       ---------------------------------------------------------
%\makecvtitle

\section{Publications}

$*$ indicates equal contribution, $\dagger$ indicates corresponding author

\subsection{Pre-prints}
    \begin{enumerate}
        \item Sebastian Guelfi$^{*}$, Karishma D’Sa$^{*}$, Juan Botía$^{*}$, Jana Vandrovcova, Regina H. Reynolds, David Zhang, Daniah Trabzuni, \textbf{Leonardo Collado-Torres}, Andrew Thomason, Pedro Quijada Leyton, Sarah A. Gagliano, Mike A. Nalls, UK Brain Expression Consortium, Kerrin S. Small, Colin Smith, Adaikalavan Ramasamy, John Hardy, Michael E. Weale$\dagger$, Mina Ryten$\dagger$. Regulatory sites for known and novel splicing in human basal ganglia are enriched for disease-relevant information. \emph{bioRxiv} 591156 (2019). doi: \httplink[10.1101/591156]{doi.org/10.1101/591156}.
        
        \item Carrie Wright$^{*}$, Anandita Rajpurohit$^{*}$, Emily E. Burke, Courtney Williams, \textbf{Leonardo Collado-Torres}, Martha Kimos, Nicholas J. Brandon, Alan J. Cross, Andrew E. Jaffe, Daniel R. Weinberger$\dagger$, Joo Heon Shin$\dagger$. Comprehensive assessment of multiple biases in small RNA sequencing reveals significant differences in the performance of widely used methods. \emph{bioRxiv} 445437 (2018). doi: \httplink[10.1101/445437]{doi.org/10.1101/445437}.
        
        \item Amanda J. Price$^{*}$, \textbf{Leonardo Collado-Torres}$^{*}$, Nikolay A. Ivanov, Wei Xia, Emily E. Burke, Joo Heon Shin, Ran Tao, Liang Ma, Yankai Jia, Thomas M. Hyde, Joel E. Kleinman, Daniel R. Weinberger, Andrew E Jaffe. Divergent neuronal DNA methylation patterns across human cortical development: Critical periods and a unique role of CpH methylation. \emph{bioRxiv} 428391 (2018). doi: \httplink[10.1101/428391]{doi.org/10.1101/428391}.
        
        \item \textbf{Leonardo Collado-Torres}, Emily E Burke, Amy Peterson, Joo Heon Shin, Richard E Straub, Anandita Rajpurohit, Stephen A Semick, William S Ulrich, BrainSeq Consortium, Cristian Valencia, Ran Tao, Amy Deep-Soboslay, Thomas M Hyde, Joel E Kleinman, Daniel R Weinberger$\dagger$, Andrew E Jaffe$\dagger$. Regional heterogeneity in gene expression, regulation and coherence in hippocampus and dorsolateral prefrontal cortex across development and in schizophrenia. \emph{bioRxiv} 426213 (2018). doi: \httplink[10.1101/426213]{doi.org/10.1101/426213}.
        
        \item Emily E Burke$^{*}$, Joshua G Chenoweth$^{*}$, Joo Heon Shin, \textbf{Leonardo Collado-Torres}, Suel Kee Kim, Nicola Micali, Yanhong Wang, Richard E Straub, Daniel J Hoeppner, Huei-Ying Chen, Alana Lescure, Kamel Shibbani, Gregory R Hamersky, BaDoi N Phan, William S Ulrich, Cristian Valencia, Amritha Jaishankar, Amanda J Price, Anandita Rajpurohit, Stephen A Semick, Roland Bürli, James C Barrow, Daniel J Hiler, Stephanie Cerceo Page, Keri Martinowich, Thomas M Hyde, Joel E Kleinman, Karen F Berman, José A Apud, Alan J Cross, Nick J Brandon, Daniel R Weinberger, Brady J Maher, Ronald DG McKay$\dagger$, Andrew E Jaffe$\dagger$. Dissecting transcriptomic signatures of neuronal differentiation and maturation using iPSCs. \emph{bioRxiv} 380758 (2018). doi: \httplink[10.1101/380758]{doi.org/10.1101/380758}.
        
        \item Fu J, Kammers K, Nellore A, \textbf{Collado-Torres L}, Leek JT, Taub MA. RNA-seq transcript quantification from reduced-representation data in recount2. \emph{bioRxiv} 247346 (2018). doi: \httplink[10.1101/247346]{doi.org/10.1101/1247346}.
    \end{enumerate}

\subsection{Peer-reviewed}
    \begin{enumerate}
        
        \item Stephen A Semick, Rahul A Bharadwaj, \textbf{Leonardo Collado-Torres}, Ran Tao, Joo Heon Shin, Amy Deep-Soboslay, James R. Weiss, Daniel R Weinberger, Thomas M Hyde, Joel E Kleinman, Andrew E Jaffe$\dagger$, Venkata S Mattay$\dagger$. Integrated DNA methylation and gene expression profiling across multiple brain regions implicate novel genes in Alzheimer’s disease. \emph{Acta Neuropathologica} 2019. doi: \httplink[10.1007/s00401-019-01966-5]{doi.org/10.1007/s00401-019-01966-5}
        \\ Pre-print: \emph{bioRxiv} 430603 (2018). doi: \httplink[10.1101/430603]{doi.org/10.1101/430603}.
            
        \item Helena Kuri-Magaña, \textbf{Leonardo Collado-Torres}, Andrew E Jaffe, Humberto Valdovinos-Torres, Marbella Ovilla-Muñoz, Juan M Téllez-Sosa, Laura C Bonifaz-Alfonzo, Jesús Martínez-Barnetche. Non-coding Class Switch Recombination-Related Transcription in Human Normal and Pathological Immune Responses. \emph{Frontiers in Immunology} 2018. doi: \httplink[10.3389/fimmu.2018.02679]{doi.org/10.3389/fimmu.2018.02679}
        \\ Pre-print: \emph{bioRxiv} 384172 (2018). doi: \httplink[10.1101/384172]{doi.org/10.1101/384172}.
        
        \item Semick SA, \textbf{Collado-Torres L}, Markunas CA, Shin JH, Deep-Soboslay A, Tao R, Huestis MA, Bierut LJ, Maher BS, Johnson EO, Hyde TM, Weinberger DR, Hancock DB, Kleinman JE$\dagger$, Jaffe AE$\dagger$. Developmental effects of maternal smoking during pregnancy on the human frontal cortex transcriptome. \emph{Molecular Psychiatry} 2018. doi: \httplink[110.1038/s41380-018-0223-1]{doi.org/10.1038/s41380-018-0223-1}
        \\ Pre-print: \emph{bioRxiv} 236968 (2017). doi: \httplink[10.1101/236968]{doi.org/10.1101/236968}.
        
        \item Jaffe AE, Straub R, Shin JH, Tao R, Gao Y, \textbf{Collado-Torres L}, Kam-Thong T, Xi HS, Quan J, Chen Q, Colantuoni C, Ulrich WS, Maher BJ, Deep-Soboslay A, The BrainSeq Consortium, Cross AJ, Brandon NJ, Leek JT, Hyde TM, Kleinman JE, Weinberger DR. Developmental and genetic regulation of the human cortex transcriptome illuminate schizophrenia pathogenesis. \emph{Nat. Neurosci.} 2018. doi: \httplink[10.1038/s41593-018-0197-y]{doi.org/10.1038/s41593-018-0197-y}.
        \\ Pre-print: \emph{bioRxiv} 124321 (2017). doi: \httplink[10.1101/145656]{doi.org/10.1101/124321}.
        
        \item Ellis SE, \textbf{Collado-Torres L}, Jaffe AE, Leek JT. Improving the value of public RNA-seq expression data by phenotype prediction. \emph{Nucl. Acids Res.} 2018. doi: \httplink[10.1093/nar/gky102]{doi.org/10.1093/nar/gky102}.
        \\ Pre-print: \emph{bioRxiv} 145656 (2017). doi: \httplink[10.1101/145656]{doi.org/10.1101/145656}.
        
        \item \textbf{Collado-Torres L}$\dagger$, Nellore A, Jaffe AE. 
        recount workflow: Accessing over 70,000 human RNA-seq samples with Bioconductor [version 1; referees: 1 approved, 2 approved with reservations]. \emph{F1000Research} (2017). doi: \httplink[10.12688/f1000research.12223.1]{doi.org/10.12688/f1000research.12223.1}.
        \\ Winning entry for the \httplink[Bioinformatics Peer Prize III]{bioinformatics-peer-prize-iii.thinkable.org}.
        
        \item Wright C, Shin JH, Rajpurohit A, Deep-Soboslay A, \textbf{Collado-Torres L}, Brandon NJ, Hyde TM, Kleinman JE, Jaffe AE, Cross AJ, Weinberger DR. Altered expression of histamine signaling genes in autism spectrum disorder. \emph{Translational Psychiatry} 2017. doi: \httplink[10.1038/tp.2017.87]{doi.org/10.1038/tp.2017.87}.
        
        \item \textbf{Collado-Torres L}$^{*}$, Nellore A$^{*}$, Kammers K, Ellis SE, Taub MA, Hansen KD, Jaffe AE, Langmead B, Leek JT. Reproducible RNA-seq analysis using \emph{recount2}. \emph{Nature Biotechnology} 2017. doi: \httplink[10.1038/nbt.3838]{doi.org/10.1038/nbt.3838}.
        \\ Pre-print: \emph{bioRxiv} 068478 (2016). doi: \httplink[10.1101/068478]{doi.org/10.1101/068478}.
        
        \item Nellore A, Jaffe AE, Fortin JP, Alquicira-Hernández J, \textbf{Collado-Torres L}, Wang S, Phillips RA, Karbhari N, Hansen KD, Langmead B, Leek JT. Human splicing diversity and the extent of unannotated splice junctions across human RNA-seq samples on the Sequence Read Archive. \emph{Genome Biology} 2016. doi: \httplink[10.1186/s13059-016-1118-6]{doi.org/10.1186/s13059-016-1118-6}.
        \\ Pre-print: \emph{bioRxiv} 038224 (2016). doi: \httplink[10.1101/038224]{doi.org/10.1101/038224}.
        
        \item \textbf{Collado-Torres L}, Nellore A, Frazee AC, Wilks C, Love MI, Langmead B, Irizarry RA, Leek JT, Jaffe AE. Flexible expressed region analysis for RNA-seq with derfinder. \emph{Nucl. Acids Res.} 2016. doi: \httplink[10.1093/nar/gkw852]{doi.org/10.1093/nar/gkw852}.
        \\ Pre-print: \emph{bioRxiv} 015370 (2016). doi: \httplink[10.1101/015370]{doi.org/10.1101/015370}.
        
        \item Nellore A, \textbf{Collado-Torres L}, Jaffe AE, Alquicira-Hernández J, Wilks C, Pritt J, Morton J, Leek JT, Langmead B. Rail-RNA: Scalable analysis of RNA-seq splicing and coverage. \emph{Bioinformatics} 2016. doi: \httplink[10.1093/bioinformatics/btw575]{doi.org/10.1093/bioinformatics/btw575}.
        \\ Pre-print: \emph{bioRxiv} 019067 (2015). doi: \httplink[10.1101/019067]{doi.org/10.1101/019067}.
        
        \item \textbf{Collado-Torres L}, Jaffe AE and Leek JT. regionReport: Interactive reports for region-level and feature-level genomic analyses [version2; referees: 2 approved, 1 approved with reservations]. \emph{F1000Research} 2016, 4:105. doi: \httplink[10.12688/f1000research.6379.2]{doi.org/10.12688/f1000research.6379.2}.
        \\ Pre-print: \emph{bioRxiv} 016659 (2015). doi: \httplink[10.1101/016659]{doi.org/10.1101/016659}.
        
        \item Jaffe AE, Shin J, \textbf{Collado-Torres L}, Leek JT, et al. Developmental regulation of human cortex transcription and its clinical relevance at single base resolution. \emph{Nat. Neurosci.} 2015. doi: \httplink[10.1038/nn.3898]{doi.org/10.1038/nn.3898}.
        
        \item Shank EA, Klepac-Ceraj V, \textbf{Collado-Torres L}, Powers GE, Losick R, Kolter R. Interspecies interactions that result in Bacillus subtilis forming biofilms are mediated mainly by members of its own genus. \emph{Proc. Natl. Acad. Sci.} U.S.A. 2011 Nov;108(48):E1236–1243. doi: \httplink[10.1073/pnas.1103630108]{doi.org/10.1073/pnas.1103630108}.
        
        \item Gama-Castro S, Salgado H, Peralta-Gil M, Santos-Zavaleta A, Muñiz-Rascado L, Solano-Lira H, Jimenez-Jacinto V, Weiss V, Garc\'ia-Sotelo JS, L\'opez-Fuentes A, Porr\'on-Sotelo L, Alquicira-Hern\'andez S, Medina-Rivera A, Mart\'inez-Flores I, Alquicira-Hern\'andez K, Mart\'inez-Adame R, Bonavides-Mart\'inez C, Miranda-R\'ios J, Huerta AM, Mendoza-Vargas A, \textbf{Collado-Torres L}, Taboada B, Vega-Alvarado L, Olvera M, Olvera L, Grande R, Morett E, Collado-Vides J. RegulonDB version 7.0: transcriptional regulation of Escherichia coli K-12 integrated within genetic sensory response units (Gensor Units). \emph{Nucleic Acids Res.} 2011 Jan;39(Database issue):D98–105. doi: \httplink[10.1093/nar/gkq1110]{doi.org/10.1093/nar/gkq1110}.
    \end{enumerate}
\subsection{Books}
    \begin{enumerate}
        \item Frazee AC, \textbf{Collado-Torres L}, Jaffe AE, Langmead B, Leek JT. Measurement, Summary, and Methodological Variation in RNA-sequencing in Statistical Analysis of Next Generation Sequencing Data, \emph{Springer}, 2014, 115-128.
    \end{enumerate}


\subsection{In prep}
\cvitem{2015}{\textbf{Collado-Torres L}, Jaffe AE, Leek JT. regionReport: Interactive reports for annotation agnostic RNA-seq differential expression analyses}

\end{document}


%% end of file `template.tex'.


